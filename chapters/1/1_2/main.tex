\section{Hiện trạng nghiệp vụ}
\subsection{Quy trình nghiệp vụ}
Các quy trình nghiệp vụ bao gồm:
\begin{itemize}
    \item Quản lý bệnh nhân
    \begin{itemize}
        \item Tiếp nhận bệnh nhân mới.
        \item Lưu trữ thông tin bệnh nhân (hồ sơ cá nhân, tiền sử bệnh).
        \item Theo dõi tình trạng sức khỏe của bệnh nhân.
    \end{itemize}
    \item Quản lý lịch hẹn khám bệnh
    \begin{itemize}
        \item Bệnh nhân được bác sĩ hẹn khám trực tiếp.
        \item Nhân viên y tế được sắp xếp lịch khám với bác sĩ để hỗ trợ.
    \end{itemize}
    \item Khám chữa bệnh
    \begin{itemize}
        \item Bệnh nhân đến khám theo lịch.
        \item Bác sĩ kiểm tra triệu chứng, yêu cầu xét nghiệm nếu cần.
        \item Đưa ra chẩn đoán và kê đơn thuốc.
    \end{itemize}
    \item Quản lý xét nghiệm và chẩn đoán hình ảnh
    \begin{itemize}
        \item Bác sĩ chỉ định xét nghiệm, chụp X-quang, MRI,…
        \item Nhân viên y tế thực hiện việc xét nghiệm, chụp X-quang, …
        \item Kết quả sẽ được trả về cho bác sĩ và bệnh nhân.
    \end{itemize}
    \item Quản lý phẫu thuật và điều trị nội trú
    \begin{itemize}
        \item Tiếp nhận bệnh nhân cần phẫu thuật hoặc điều trị nội trú.
        \item Sắp xếp bác sĩ, y tá phụ trách.
        \item Theo dõi bệnh nhân trong quá trình điều trị.
    \end{itemize}
    \item Quản lý đơn thuốc
    \begin{itemize}
        \item Bác sĩ là người kê đơn thuốc.
        \item Bệnh nhân có thể nhận thuốc từ nhà thuốc nội bộ hoặc nhà thuốc xung quanh
        ngoài bệnh viện.
        \item Theo dõi quá trình sử dụng thuốc để đánh giá hiệu quả của thuốc.
    \end{itemize}
    \item Quản lý nhân sự y tế (bác sĩ, y tá, điều dưỡng,…)
    \begin{itemize}
        \item Lưu trữ thông tin cá nhân của nhân sự.
        \item Quản lý lịch làm việc của nhân viên.
    \end{itemize}
    \item Quản lý tài chính, bảo hiểm y tế
    \begin{itemize}
        \item Xác nhận bệnh nhân có bảo hiểm hay không.
        \item Tính toán chi phí điều trị và thanh toán.
        \item Hoàn tất các thủ tục liên quan đến bảo hiểm y tế.
    \end{itemize}
\end{itemize}

\subsection{Cách thực hiện nghiệp vụ}

\begin{centerpage}    
\begin{tabular}{|>{\raggedright}p{3cm}|p{6cm}|p{3.2cm}|}
    \hline
    \textbf{Nghiệp vụ} & \textbf{Các Công Đoạn Chính} & \textbf{Bộ Phận Liên Quan} \\
    \hline
    Quản lý bệnh nhân & Đăng ký thông tin bệnh nhân → Lưu hồ sơ → Cập nhật thông tin  & Nhân viên tiếp nhận, bác sĩ \\
    \hline
    Quản lý lịch hẹn & Bệnh nhân đăng ký lịch → Xác nhận lịch → Thông báo nhắc lịch & Nhân viên hành chính, bác sĩ \\
    \hline 
    Khám chữa bệnh & Tiếp nhận bệnh nhân → Kiểm tra triệu chứng → Chẩn đoán → Kê đơn thuốc & Bác sĩ, y tá \\
    \hline
    Quản lý xét nghiệm và chuẩn đoán hình ảnh & Chỉ định xét nghiệm, chụp ảnh → Thực hiện xét nghiệm, chụp ảnh → Báo cáo kết quả & Bác sĩ, kĩ thuật viên y tế \\
    \hline
    Quản lý phẫu thuật và điều trị nội trú & Đánh giá tình trạng bệnh nhân → Chuẩn bị phẫu thuật, điều trị → Theo dõi hậu phẫu, hậu điều trị & Bác sĩ, y tá, điều dưỡng \\
    \hline
    Quản lý đơn thuốc & Kê đơn → Xác nhận thuốc có sẵn → Phát thuốc & Bác sĩ, dược sĩ, bệnh nhân \\
    \hline
    Quản lý nhân sự y tế & Lưu thông tin nhân viên → Phân công công việc → Theo dõi lịch làm việc & Quản lý nhân sự, bác sĩ, nhân viên y tế\\
    \hline
    Quản lý tài chính, bảo hiểm & Xác nhận thông tin bảo hiểm → Tính toán chi phí → Thanh toán & Nhân viên tài chính \\
    \hline 
\end{tabular}
\end{centerpage}

\subsection{Tần suất, thời điểm thực hiện}

\begin{centerpage}    
\begin{tabular}{|>{\centering\arraybackslash}p{6cm}|>{\centering\arraybackslash}p{6cm}|}
\hline
\textbf{Nghiệp Vụ} & \textbf{Tần Suất} \\
\hline
Quản lý bệnh nhân & Hàng ngày \\
\hline
Quản lý lịch hẹn & Hàng ngày \\
\hline
Khám chữa bệnh & Hàng ngày \\
\hline
Quản lý xét nghiệm và chuẩn đoán hình ảnh & Theo chỉ định của bác sĩ \\
\hline
Quản lý phẫu thuật và điều trị nội trú & Theo lịch phẫu thuật \\
\hline
Quản lý đơn thuốc & Khi có bệnh nhân mới hoặc thay đổi toa thuốc \\
\hline
Quản lý nhân sự y tế & Hàng tháng (cập nhật lịch làm việc) \\
\hline
Quản lý tài chính, bảo hiểm & Khi bệnh nhân thanh toán hoặc cần xác nhận bảo hiểm \\
\hline
\end{tabular}
\end{centerpage}


\subsection{Khối lượng tác vụ và quyết định}

\begin{centerpage}    
\begin{tabular}{|>{\centering\arraybackslash}p{6cm}|>{\centering\arraybackslash}p{6cm}|}
\hline
\textbf{Nghiệp Vụ} & \textbf{Khối Lượng Xử Lý} \\
\hline
Quản lý bệnh nhân & 100 - 500 hồ sơ/ngày \\
\hline
Quản lý lịch hẹn & 200 - 1000 lịch hẹn/ngày \\
\hline
Khám chữa bệnh & 100 - 500 ca/ngày \\
\hline
Quản lý xét nghiệm và chuẩn đoán hình ảnh & 50 - 300 xét nghiệm/ngày \\
\hline
Quản lý phẫu thuật và điều trị nội trú & 10 - 50 ca/ngày \\
\hline
Quản lý đơn thuốc & 100 - 500 đơn/ngày \\
\hline
Quản lý tài chính, bảo hiểm & 100 - 500 thanh toán/ngày \\
\hline
\end{tabular}
\end{centerpage}

\subsection{Đánh giá nghiệp vụ hiện tại}

\begin{itemize}
    \item \textbf{Ưu điểm} 
    \begin{itemize}
        \item Đã có quy trình rõ ràng trong việc tiếp nhận, khám bệnh, và điều trị.
        \item Nhân viên có kinh nghiệm, có thể xử lý tốt các tình huống.
        \item Có hợp tác với bảo hiểm y tế để hỗ trợ bệnh nhân.
    \end{itemize}

    \item \textbf{Nhược điểm}
    \begin{itemize}
        \item Quản lý hồ sơ bệnh án vẫn chủ yếu là giấy tờ, gây khó khăn khi tra cứu.
        \item Quy trình đặt lịch hẹn còn thủ công, làm việc quản lí trở nên rất kém hiệu quả.
        \item Bệnh nhân khó theo dõi lịch sử khám bệnh và đơn thuốc của mình.
        \item Bác sĩ khó nắm bắt tình hình bệnh nhân đang thưc hiện chữa trị tại nhà.
        \item Chưa có hệ thống cảnh báo tình trạng khẩn cấp của bệnh nhân.
        \item Kết quả xét nghiệm và phân tích hình ảnh chưa liên kết chặt chẽ với bệnh án.
    \end{itemize}
\end{itemize}
